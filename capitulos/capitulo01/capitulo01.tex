%*******************************************************
% Capitulo uno
%*******************************************************

\chapter{Introducción}

Este documento expone la propuesta de creación de un modelo de negocio basado en una aplicación móvil de valoración y comparación de marcas. El documento está construido usando la especificación Archimate 2.0 \cite{iacob2012archimate} y el Modelo Canvas \cite{osterwalder2013business}\cite{osterwalder2004business}. En el presente capítulo se detallan elementos básicos como el resumen de la propuesta, los referentes que dan origen a la idea y el alcance del documento.

\section{Propuesta}

Se propone un modelo de valoración independiente y colectivo de marcas por medio de una aplicación móvil de descarga gratuita, que además permite retroalimentar a la marca acerca de situaciones específicas sobre atención, calidad de producto o situaciones anormales. Además la aplicación explorará las redes sociales para analizar y publicar información relevante a las marcas.

Dado que cada marca puede representar elementos valorables diferentes como empresas, productos y servicios, se propone la calificación de elementos de manera independiente. Cada administrador de marca podrá controlar qué elementos podrán ser expuestos al consumidor con el fin de organizar los elementos disponibles, de tal manera que cada empresa podrá proponer modificaciones a la base de datos para reunir o especificar sus productos/servicios o definir métricas diferentes a la estándar. Se propone el uso de una base de datos inicial de marcas públicas y privadas.

El modelo de negocio se basa en ads de publicidad para el usuario final, suscripción premium para las empresas en la aplicación y en servicios conexos para empresas que pertenezcan a la base de datos. Entre los servicios ofrecidos pueden estar: Estudios de mercado, publicidad, reestructuración de procesos de atención y servicio al cliente basados en modelos digitales o comunicación online con el cliente.

\section{Antecedentes y referentes}

\subsection{Ejercicio de comparación: Apps de taxis}

El primer referente para la concepción de ésta idea de negocio es la comparación de atributos competitivos en apps de solicitud de taxi en Colombia. Un rápido repaso por cada una de las aplicaciones disponibles en el mercado muestra que UBER \cite{avalos2015baby} vence en diferentes aspectos a otras aplicaciones resaltando un atributo en particular: la calificación al conductor. 

Los esquemas de valoración tienen efectos de control y aseguramiento de la calidad porque permiten tomar acciones sobre los prestadores. Los mismos efectos de valoración se pueden ver en organizaciones que han implementado medidas de valoración que transforman calificaciones en medidas de tracción de usuarios. En el caso de Uber los conductores valorados por debajo de 4.0 reciben sanciones que impiden que se vuelvan a presentar futuros inconvenientes, ya sea porque se corrige el comportamiento del conductor o se retira de la plataforma.

\subsection{Aplicaciones de valoración de nicho específico}

El segundo referente importante son las aplicaciones de valoración de servicios en nichos específicos. Por ejemplo, en el sector del turismo y similares se encuentran aplicaciones con diferentes alcances como \textit{tripadvisor} o \textit{foursquare}. Cada una de ellas comparte el mismo modelo encontrado en el primer referente: la valoración por el usuario para la marca. En el sector educativo se pueden encontrar aplicaciones como SchoolMars, cuyo epígrafe es \textit{"Ponle nota a tu colegio"}\cite{schollmars:2013:Online}. 
Para cada nicho de mercado se pueden encontrar aplicaciones, unas más populares que otras, que permiten construir conocimiento colectivo de valoración empresarial de productos o servicios.

En latinoamérica existen los siguientes referentes que pueden ser consultados: \textit{http://www.apontador.com.br/} lider en Brasil en la valoración de lugares de entretenimiento y \textit{https://kekanto.com.co/} También una empresa brasilera pero con presencia en Colombia, y \textit{http://www.guiatodo.com.co/} empresa colombiana que reúne información de sitios de turismo.

\subsection{Redes sociales}

El tercer referente es el uso de las redes sociales como canal de expresión de las valoraciones y percepciones de marca. Todos los días Twitter, Facebook, Instagram, Snapchat, son usadas por usuarios para manifestar inconformidad con algún producto o servicio. La información publicada  reemplaza al tradicional voz a voz y con quejas de alto tono se logra difusión viral de la molestia. 

Las marcas han tenido que enfrentar los nuevos canales mediante community managers y otros cargos que antes no existían. La estrategia actual es hacer contrapeso a las publicaciones con impactos positivos de marca y disuación de comentarios. Aunque ésta información es muy relevante se pierde en el flujo natural de las redes y en cuestión de horas son publicaciones del pasado. Es importante notar que en este momento no se está capitalizando la información distribuida en las redes sociales.

\subsection{Aplicaciones de valoración global}

Los referentes más cercanos a la presente idea de negocio son \textit{Google Reviews} y \textit{yelp.com}\cite{luca2011reviews}. Google Reviews es una plataforma que permite a los usuarios realizar reseñas para locales comerciales o empresas con el fin de entregar a los próximos usuarios una valoración que es útil para tomar decisiones basadas en información construida por los usuarios de las marcas que también usan Google. 

\textit{Yelp.com} es un sitio web y aplicación movíl donde los consumidores pueden dejar revisiones y comentarios de restaurantes y otros comercios. Fue fundada en 2004, en San Francisco. Para el año 2011 contenían más de 10 millones de revisiones y recibía mas de 40 millones de visitantes únicos por mes en estados unidos. Actualmente no tienen presencia en los países de suramérica.

\section{Alcance y limitaciones}

Este documento presenta una descripción arquitectónica por medio de los puntos de vista de la especificación Archimate 2.0. incluyendo capa motivacional, negocio, aplicación e infraestructura. No se describe ningún aspecto de la viabilidad financiera. Para la descripción arquitectónica se usa una construcción básica del negocio obtenida a partir del modelo canvas. 

Las siguientes partes del documento describen aspectos básicos de la metodología y el lenguaje usado en el documento, la información obtenida después del análisis bajo la propuesta Business Canvas Model\cite{osterwalder2004business} y en la tercera parte del documento cada una de los puntos de vista de Archimate 2.0. Al final algunas conclusiones y trabajo futuro para mejorar ésta presentación. 

%*******************************************************
% Capitulo dos ///// SE Modifica para el capitulo uno
%*******************************************************

\section{Metodología}

El proceso de especificación del modelo de negocio tiene tres etapas principales:
\begin{itemize}
    \item Recolección de referentes fundamentales y examen de la competencia (documentado en el Capítulo 1).
    \item Análisis de modelo de negocio con la estructura Business Canvas Model \cite{osterwalder2004business} y descripción de los nueve cuadrantes.
    \item Descripción avanzada de la propuesta usando los 15 puntos de vista de la especificación Archimate 2.0.
\end{itemize}

El proceso de construcción pasa por la especificación de cuatro capas, a saber: Capa Motivacional, capa de negocio, capa de aplicación y capa de infraestructura.


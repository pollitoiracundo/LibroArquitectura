%*******************************************************
% Capitulo dos
%*******************************************************

\chapter{Lenguaje usado}

La descripción de los componentes de arquitectura empresarial y sus relaciones requieren un lenguaje de modelado que guíe de manera elocuente la exposición de todos los detalles del negocio.

Éste documento usa la especificación de modelado arquitectónico ArchiMate 2.0. \cite{iacob2012archimate} para facilitar la exposición de la presente propuesta. El diseño del lenguaje ArchiMate inicia desde un conjunto de conceptos genéricos como entidad y relación y agrega a especificaciones anteriores, como UML, una gran variedad de conceptos específicos que derivan de procesos y aplicaciones de procesos para después aplicar conceptos específicos de dominio y arquitectura empresarial. 

A continuación se expone brevemente los detalles más importantes del estándar de modelado.

\section{Conceptos principales}

Los tres elementos principales del lenguaje, que representan entidades del mundo real, son: Elementos estructurales activos, Elementos de comportamiento y Sujetos que expresan dicho comportamiento o en otras palabras objetos.

\begin{itemize}
    \item Un \textit{elemento estructural activo} es definido como una entidad capaz de ejecutar un comportamiento.
    \item Un \textit{elemento de comportamiento} es una unidad ejecutada por uno o más elementos estructurales activos.
    \item Los \textit{elementos estructurales pasivos} son objetos en los que un comportamiento es ejecutado.
\end{itemize}

Éstos tres aspectos son inspirados en frases con sujetos (estructuras activas), verbos (comportamientos) y objetos (estructuras pasivas).

Por otro lado, se realiza una distinción entre la vista interna y externa de un sistema y estas se conectan. Bajo esta conexión se aclara también el termino de interfaz.

\begin{itemize}
    \item Un \textit{servicio} es una unidad de funcionalidad expone un sistema a su ambiente mientras oculta operaciones internas que entregan valor.
\end{itemize}

Dado que el servicio es el comportamiento visible externo de un sistema, existe un entorno en el que el sistema es usado a través del servicio. El servicio existe por una motivación que debe ser entendida como el valor del sistema o el servicio. Para los que usan de manera externa la combinación del servicio y su valor sólo no es relevante la información interna del sistema. Los servicios son accequibles a través de interfaces, las cuales son la vista externa de un elemento activo.

\begin{itemize}
    \item Una \textit{interfaz} es definida como un punto de acceso para uno o más servicios y esta disponible al entorno del sistema.
\end{itemize}

\section{Colaboración e interacción}

En ArchiMate se puede distinguir entre un comportamiento realizado por un elemento estructural o un comportamiento colectivo (interacción) que es ejecutado por una colaboración de múltiples elementos estructurales.

\begin{itemize}
    \item Una \textit{colaboración} es definida como una agrupación de dos o más elementos estructurales que desarrollan un comportamiento colectivo.
\end{itemize}

El comportamiento colectivo puede ser modelado como una interacción.

\begin{itemize}
    \item Una \textit{interacción} es definida como una unidad de comportamiento ejecutada por una colaboración de dos o más elementos estructurales.
\end{itemize}

\section{Relaciones y capas}

Para relacionar las entidades, ArchiMate cuenta con una serie de relaciones análogas a otros lenguajes gráficos de modelado como UML o BPMN.

También cuenta con tres capas principales basadas en especializaciones de los conceptos básicos. 

\begin{enumerate}

\item La capa de negocio ofrece productos y servicios a clientes  externos, los cuales son realizados en la organización por procesos de negocio y actores del negocio.

\item La capa de aplicación soporta el negocio con servicios de aplicación que son realizados por aplicaciones (software).

\item La capa de tecnología (Infraestructura) ofrece servicios de infraestructura (por ejemplo: procesamiento, almacenamiento y servicios de comunicación) usados para ejecutar aplicaciones realizadas por computadores, hardware de comunicación y sistemas de software.

\end{enumerate}




%*******************************************************
% Capitulo tres
%*******************************************************

\chapter{Modelo Canvas}

La propuesta para describir modelos de negocio presentada por Osterwalder en \cite{osterwalder2004business} muestra un esquema en donde nueve módulos básicos reflejan la lógica que sigue una empresa para conseguir ingresos. Estos nueve módulos cubren las cuatro áreas principales de un negocio: clientes, oferta, infraestructuras y viabilidad económica.

El Modelo Canvas describe una composición de modelo de negocio donde la propuesta de valor (que hace parte del producto junto con la oferta) ocupa una posición central. Esta tiene que ser llevada a unos segmentos de clientes con los que se establecerán unas relaciones. Para hacerlo hay que describir unos canales específicos. Internamente, se deben desarrollar unas actividades clave con unos recursos clave. Para lograrlo se interactúa con unos Aliados importantes. 

Finalmente se especifican los elementos que permiten sustentar la operación en términos de costos y las fuentes de ingreso del modelo de negocio.

\section{Aliados clave}

\textbf{¿Quienes son sus aliados clave y proveedores?}

Los aliados clave del negocio se dividen en los diferentes proveedores de tecnología que sustentan la operación y los aliados estratégicos que complementan el portafolio de servicios. Entre ellos se encuentran:

\begin{itemize}
    \item Proveedor para colocación de Ads (publicidad online) de terceros dentro de la aplicación.
    \item Proveedor para colocación de Ads en portales de internet para promocionar la aplicación.
    \item Proveedor de cloud para el despliegue de los servicios.
    \item Aliado de desarrollo de servicios de software.
    \item Aliado de consultoría en servicios de marca y publicidad.
  
\end{itemize}

\section{Actividades clave}

\textbf{¿Cuáles son las actividades más importantes en la ejecución de sus propuestas de valor?}

Las actividades principales describen aspectos de la operación y la gestión comercial. Entre ellos están:

\begin{itemize}
    \item Activación de nuevas cuentas empresariales.
    \item Procesos de servicio y consultoría sobre las cuentas.
    \item Despliegue de aplicaciones.
    \item Innovación y desarrollo en requisitos funcionales de la app.
    \item Gestión de conocimiento y análisis de datos.
    \item Tracción de nuevos usuarios finales.
\end{itemize}   

\section{Recursos clave}

\textbf{¿Cuáles son los recursos clave en la ejecución de las propuestas de valor?}

Dada la naturaleza del negocio, los recursos clave están enfocados tanto en talento humano como en la tecnología que permite la operación. Entre ellos está: 

\begin{itemize}
    \item Los desarrolladores de la aplicación.
    \item Los ambientes de desarrollo de software, pruebas y producción.
    \item La información local y externa acerca de las marcas.
    \item Los consultores en temas de marcas y en temas de atención y servicio al cliente
\end{itemize}   

\section{Propuestas de valor}

\textbf{¿Qué valor se le entregará a los clientes? ¿Qué problemas se les está ayudando a resolver?}

La propuesta de valor es el elemento más importante de la construcción del Modelo Canvas y puede ser entendida como la declaración de los beneficios que son entregados por el negocio. En este caso los externos que reciben éstos beneficios se dividen en dos. 

A la sociedad en general representada en los usuarios de la aplicación se le ofrece,
\begin{itemize}
    \item La oportunidad de tomar decisiones acertadas de relacionamiento frente a una marca basada en experiencias anteriores y sus métricas de cambio.
    \item Un espacio colectivo e independiente para que su voz sea escuchada en eventos relacionados con algún producto servicio o empresa.
\end{itemize}   

Al universo empresarial representado en las marcas con presencia en la aplicación, 
\begin{itemize}
    \item Organizar la información online de percepción de las marcas.
\end{itemize}

\section{Relaciones con los clientes}

\textbf{¿Qué tipo de relaciones con cada uno de los segmentos de clientes se espera establecer y mantener?}

Este elemento describe la relación que el negocio establece con un segmento de clientes. Una relación está basada sobre el valor (adquisición, retención, venta agregada, etc.) del cliente y en describir sus mecanismos de relacionamiento. Las relaciones con los clientes promueven la proposición de valor y son mantenidas con todo un segmento.

Cada uno de los mecanismos inicialmente planteados por los que se cumple el relacionamiento se describe a continuación.

\begin{itemize}
    \item Campaña sobre el concepto de queja con intervención en espacios físicos y espacios digitales de redes sociales. \textit{Adquisición}. \textit{Usuarios finales}.
    \item Campaña de -A mi también me paso- para construcción de voz a voz digital. \textit{Adquisición}. \textit{Usuarios finales}.
    \item Campaña sobre el concepto de calidad de servicio en espacios digitales de redes sociales. \textit{Adquisición}. \textit{Marcas}.
    \item Programa de comunicación frecuente con estadísticas de interés. \textit{Retención}. \textit{Usuarios finales} y \textit{Marcas}.   
    \item Característica funcional de -Pregúntale al usuario- para generar conversación. \textit{Retención}. \textit{Usuarios finales}.
    \item Portal de mejoramiento continuo en atención y servicio. \textit{Venta agregada}. \textit{Marcas}.
    \item \textbf{Aplicación móvil}. Elemento de relación funcional principal. \textit{Retención}. \textit{Usuarios finales}.
    \item \textbf{Portal de administración de marca.} \textit{Venta agregada}. \textit{Marcas}.
    
\end{itemize}

\section{Segmentos de clientes}

\textbf{¿Cuáles son los segmentos de clientes y usuarios?}

Se tendrá relación con dos tipos de clientes diferentes: Los usuarios finales de la aplicación, los cuales la descargarán de las tiendas de app y usarán regularmente en la valoración, que estarán en la clasificación de usuarios del milenio (millenials); y las marcas valoradas a través de la aplicación, las cuales serán empresas cuya percepción de marca pueda ser recolectada de manera digital. Resumiendo:

\begin{itemize}
    \item \textit{Usuarios finales}. En el segmento de los usuarios del milenio. 
    \item \textit{Empresas}. Pueden ser medianas y grandes de cualquier segmento de mercado que usen o estén interesadas en usar relacionamiento digital con sus usuarios.
\end{itemize}

En cualquier momento se podrá clasificar a los usuarios/marcas como activos e inactivos. Si bien se puede lograr adquisición de los clientes y usuarios podría no cumplirse la retención para algunos. Aquellos en los que se sufra abandono se considerarán como inactivos.

\section{Canales}

\textbf{¿Cuáles son las rutas de acceso o canales para acceder a los clientes?}

Indiscutiblemente los medios digitales forman parte de este modelo de negocio. La forma en la que los usuarios llegarán a las tiendas online y descargarán la aplicación dependerá de las estrategias de adquisición de clientes pero el medio importante considerado como canal serán las tiendas. Los clientes que representan las marcas accederán por el \textit{portal de administración de marca} descrito antes y a él llegarán por un website comercial que mostrará los servicios de negocio. En resumen:

\begin{itemize}
    \item \textit{Portal modelo de negocio}. Antes de registrarse, las empresas propietarias de las marcas accederán a un portal de registro y especificación de planes. 
    \item \textit{Portal de administración de marca}. Compra de otros servicios al tener acceso al modelo de administración de información de la marca.
    \item \textit{Alianzas de crecimiento vertical}. Entrar al sistema a partir de una red de portales aliados en donde se pueda hacer crecimiento vertical de clientes con producto de este modelo de negocio.
    \item \textit{Convenios empresariales}. Convenios con organizaciones empresariales como Cámaras de comercio para afiliaciones masivas.
    \item \textit{Marketplaces}. Las tiendas online de descarga de aplicaciones.
\end{itemize}

\section{Estructura de costos}

\textbf{¿Cuáles son sus costos? ¿Cuáles son los costos más altos?}

Para el funcionamiento de la aplicación se requiere una infraestructura escalable que soporte los servicios a los usuarios y marcas con un \textit{ans} más que aceptable. Además, el crecimiento constante de funcionalidades sobre la aplicación exige un grupo de desarrollo de software permanente capaz de responder con rapidez a actualizaciones de sistemas operativos entre otros. La tarea de activar nuevos clientes recae sobre un departamento comercial que constantemente está buscando la manera de traer nuevas marcas. En resumen:

\begin{itemize}
    \item \textit{Infraestura cloud}. Para soportar los servicios de operación. 
    \item \textit{Equipo de desarrollo}. Para el desarrollo permamente del software.
    \item \textit{Operación de mercadeo y comercial}. Para generar el relacionamiento con los clientes.
\end{itemize}

\section{Flujos de ingresos}

\textbf{¿De qué forma de capturarán ingresos a partir de las proposiciones de valor?}

Se capturarán ingresos de varias formas diferentes, a saber: El posicionamiento deseado de ciertas compañías dentro de la aplicación, incluyendo ads y entrega de publicidad; y los servicios agregados hacia las compañías propietarias de marcas que quieran cambiar su percepción de marca. Entre los servicios agregados también existe la posibilidad de construcción de software para atención y servicio al cliente. La aplicación no tendrá costo para los usuarios finales por lo que será de descarga gratuita. Finalmente la estructura de ingresos dependerá de: 

\begin{itemize}
    \item \textit{Niveles de suscripción de empresas dueñas de marcas}. Tener acceso a administrar la información de la marca. 
    \item \textit{Posicionamiento dentro de la aplicación}. Para presentar resultados de búsqueda prioritarios a los usuarios. 
    \item \textit{Servicios agregados de posicionamiento de marca}. Consultoría específica de posicionamiento y cambio de percepción de marca.
    \item \textit{Servicios agregados en atención y servicio al cliente}. Consultoría en procesos y construcción específica de herramientas de prestación de servicio a través o fuera de la aplicación.
    \item \textit{Estudios y datos}. Consultoría y minería en datos específicos de percepción y demografía del consumidor.
\end{itemize}


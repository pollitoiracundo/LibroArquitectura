%*******************************************************
% Capitulo nueve
%*******************************************************

\chapter{Conclusiones}

De la revisión del texto se puede destacar lo siguiente: 

\begin{itemize}
    \item Se usaron dos modelos diferentes para hacer descripción de un modelo de negocio, a saber, el modelo canvas y una propuesta arquitectónica hecha con la descripción ArchiMate. Una sirve de insumo a la otra y permite la ampliación del concepto descriptivo. Ninguno de los esquemas es mejor que el otro y describen elementos complementarios. Ambos pueden ser vistos como ontologías que organizan la información de una organización o modelo de negocio.
    
    \item Ambas descripciones metodológicas son ontológicas descritas con entidades y relaciones. Archimate con estructuras basadas en capas y elementos activos, pasivos y comportamientos, y Business Canvas Model con otro tipo de ontologías como la relación entre propducto y propuesta de valor, conceptos especíoficos del ambito empresarial.
    
    \item La descripción arquitectónica con ArchiMate solo cubre parcialmente las características que se encuentran en el modelo canvas. Y viceversa, el modelo canvas cubre solo parcialemente lo que puede ser descrito con la especificación ArchiMate.
    
\end{itemize}